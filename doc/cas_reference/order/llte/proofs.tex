
\begin{theorem} \label{thm:lno_reflexive}
$\propname{RX}(\mathrm{LNO}(S,\bullet)) \Leftrightarrow \propname{IP}(S, \bullet)$
\end{theorem}

\begin{proof}

\vspace{0.5em}

Assume $\propname{IP}(S,\bullet)$
\begin{ind}
Then for all $x$ we have that $x = x \bullet x$ and therefore $x \leq x$.
\end{ind}
Hence $\propname{IP}(S,\bullet) \Rightarrow \propname{RX}(\mathrm{LNO}(S,\bullet))$

\vspace{2em}

Assume $\neg \propname{IP}(S,\bullet)$
\begin{ind}
Then there exists $x$ such that $x \neq x \bullet x$ and therefore $x \nleq x$.
\end{ind}
Hence $\neg \propname{IP}(S,\bullet) \Rightarrow \neg \propname{RX}(\mathrm{LNO}(S,\bullet))$

\end{proof}





\begin{theorem} \label{thm:lno_transitive}
If $\propname{AS}(S, \bullet)$ then
\begin{equation*}
\propname{TR}(\mathrm{LNO}(S,\bullet)) \Leftrightarrow  \propname{TRUE}
\end{equation*}
\end{theorem}

\begin{proof}

\vspace{0.5em}

Consider arbitrary $x,y,z$
\begin{ind}
Assume $x \leq y$ ($x = x \bullet y$) and $y \leq z$ ($y = y \bullet z$)

\begin{ind}
Then we have
\begin{align*}
x 	& = x \bullet y \\
	& = x \bullet (y \bullet z) \\
	& = (x \bullet y) \bullet z \\
	& = x \bullet z
\end{align*}
and so $x \leq z$.

\end{ind}
\end{ind}

Hence $\propname{TR}(\mathrm{LNO}(S,\bullet))$
\end{proof}





\begin{theorem} \label{thm:lno_antisymmetric}
If $\propname{CM}(S, \bullet)$ then
\begin{equation*}
\propname{AY}(\mathrm{LNO}(S,\bullet)) \Leftrightarrow \propname{TRUE}
\end{equation*}
\end{theorem}

\begin{proof}

Consider arbitrary $x$ and $y$ such that $x \leq y$ ($x = x \bullet y$) and $y \leq x$ ($y = y \bullet y$). 

\begin{ind}

Then we have that
\begin{align*}
x 	& = x \bullet y \\
	& = y \bullet x \\
	& = y
\end{align*}

Hence $x \leq y$ and $y \leq x$ implies $x = y$
\end{ind}

Hence $\propname{AY}(\mathrm{LNO}(S,\bullet))$

\end{proof}




\begin{theorem} \label{thm:lno_total}
If $\propname{CM}(S,\bullet)$ then
\begin{equation*}
\propname{TO}(\mathrm{LNO}(S,\bullet)) \Leftrightarrow \propname{SL}(S, \bullet)
\end{equation*}
\end{theorem}


\begin{proof}

\vspace{0.5em}

Assume $\propname{SL}(S,\bullet)$
\begin{ind}
Consider arbitrary $x$ and $y$.
\begin{ind}
Case $x = x \bullet y$
\begin{ind}
Then we have that $x \leq y$
\end{ind}

Case $y = x \bullet y$
\begin{ind}
Then we have that $y \bullet x = x \bullet y = y$ and so $y \leq x$.
\end{ind}
\end{ind}
\end{ind}
Hence $\propname{SL}(S,\bullet) \Rightarrow \propname{TO}(\mathrm{LNO}(S,\bullet))$

\vspace{2em}

Assume $\neg \propname{SL}(S,\bullet)$
\begin{ind}
Then there exists $x$ and $y$ such that $x \neq x \bullet y \neq y$. As $x \neq x \bullet y$ we have that $x \nleq y$ and as $y \neq x \bullet y = y \bullet x$ we also have that $y \nleq x$.
\end{ind}
Hence $\neg \propname{SL}(S,\bullet) \Rightarrow \neg \propname{TO}(\mathrm{LNO}(S,\bullet))$
\end{proof}





\begin{theorem} \label{thm:lno_bottom}
$\propname{BM}(\mathrm{LNO}(S,\bullet)) \Leftrightarrow \propname{LAN}(S, \bullet)$
\end{theorem}

\begin{proof}

\vspace{0.5em}

Assume $\propname{LAN}(S,\bullet)$
\begin{ind}
For all $x$ we then have $\omega = \omega \bullet x$ which shows $\omega \leq x$. Therefore $\omega$ is the required bottom element.
\end{ind}
Hence $\propname{LAN}(S,\bullet) \Rightarrow \propname{BM}(\mathrm{LNO}(S,\bullet))$

\vspace{2em}

Assume $\neg \propname{LAN}(S,\bullet)$
\begin{ind}
Consider arbitrary $x$. Then by $\neg \propname{LAN}(S,\bullet)$ there exists $y$ such that $x \neq x \bullet y$ and hence $x \nleq y$. Therefore there is no bottom element.
\end{ind}
Hence $\neg \propname{LAN}(S,\bullet) \Rightarrow \neg \propname{BM}(\mathrm{LNO}(S,\bullet))$
\end{proof}





\begin{theorem} \label{thm:lno_top}
$\propname{TP}(\mathrm{LNO}(S,\bullet)) \Leftrightarrow \propname{RID}(S, \bullet)$
\end{theorem}

\begin{proof}

\vspace{0.5em}

Assume $\propname{RID}(S,\bullet)$
\begin{ind}
For all $x$ we then have $x = x \bullet i$ which shows $x \leq i$. Therefore $i$ is the required top element.
\end{ind}
Hence $\propname{RID}(S,\bullet) \Rightarrow \propname{TP}(\mathrm{LNO}(S,\bullet))$

\vspace{2em}

Assume $\neg \propname{RID}(S,\bullet)$
\begin{ind}
Consider arbitrary $x$. Then by $\neg \propname{RID}(S,\bullet)$ there exists $y$ such that $y \neq y \bullet x$ and hence $y \nleq x$. Therefore there is no top element.
\end{ind}
Hence $\neg \propname{RID}(S,\bullet) \Rightarrow \neg \propname{TP}(\mathrm{LNO}(S,\bullet))$
\end{proof}


