

\chapter{Other Problems} 

\section{Matrix algebras}

\section{Swap}

We might want to define the following combinator for bi-semigroups:
\[ 
\mathrm{Swap}(S,\ \oplus,\ \otimes)   
   \equiv 
   (S,\ \otimes,\ \oplus). 
\] 
The disadvantage of this is that we would then have to introduce 
(and close with respect to) these dual properties for other bi-semigroup combinators: $\propname{LD}(S,\ \otimes,\ \oplus)$, $\propname{RD}(S,\ \otimes,\ \oplus)$, $\propname{LAB}(S,\ \otimes,\ \oplus)$ and $\propname{RAB}(S,\ \otimes,\ \oplus)$.

\section{$k$-best paths}

%% Assume $(S,\ \oplus,\\otims)$ with 
%% $\propname{SL}(\oplus)$ and $\propname{CM}(\oplus)$. 

%% \[
%%   \mathcal{T}_k \equiv 
%%   (\mathbb{T}_k,\  
%%   \oplus_k,     \ 
%%   \otimes_k,    \ 
%%   \overline{0}_k,\
%%   \overline{1}_k
%%   )
%% \] 
%% where 
%% \[
%% \begin{array}{rcl} 
%% (a_0,\ \dots,\ a_k) \oplus_k (b_0,\ \dots,\ b_k) 
%%    & \equiv 
%%    & \min_k (a_0,\dots,\ a_k,\ b_0,\ \dots,\ b_k) \\
%% \\ 
%% \overline{0}_k 
%%    & \equiv 
%%    & (\infty,\ \infty,\ \cdots,\ \infty)\\ 
%% \\ 
%% (a_0,\ \dots,\ a_k) \otimes_k (b_0,\ \dots,\ b_k) 
%%    & \equiv 
%%    & \min_k(a_0 + b_0,\ a_0 + b_1,\ \dots,\ a_k + b_k) \\ 
%% \\
%% \overline{1}_k 
%%    & \equiv 
%%    & (0,\ \infty,\ \cdots,\ \infty)
%% \end{array} 
%% \]
%% \end{block}




\section{Semi-direct product}

\section{General reductions}

%% $(T,\ r,\ \approx)$ is well-formed if 
%% $\approx$ is an equivalence relation over $T$ and 
%% for all $x \in T$ we have $r(x) \approx x$. 
%% \[
%%     \SemCarrier{(t,\ r,\ =)}  \equiv  (\{ x \in t \mid r(x) = x\},\ =_r)
%% \] 

%% \[
%% \begin{array}{rcl} 
%% \mathrm{Nat}  & \equiv & (\NN,\ \Id,\ =_\NN)  \\ 
%% \mathrm{Bool} & \equiv & (\BB,\ \Id,\ =_\BB)  
%% \end{array} 
%% \] 

%% \[
%% \begin{array}{rcl} 
%% \mathrm{Prod}((T_1,\ r_1,\ \approx_1),\ (T_2,\ r_2,\ \approx_2)) 
%%    & \equiv 
%%    & (T_1 \times T_2,\ r_1 \times r_2,\ \approx_1 \times \approx_2) \\ 
%% \mathrm{Sum}((T_1,\ r_1,\ \approx_1),\ (T_2,\ r_2,\ \approx_2)) 
%%    & \equiv 
%%    & (T_1 + T_2,\ r_1 + r_2,\ \approx_1 + \approx_2) \\ 
%% \mathrm{Seq}(T\ r\ \approx)
%%    & \equiv 
%%    & (\Tseq{T},\ \MapSeq{r},\ \approx_{seq}) \\ 
%% \mathrm{Set}(T\ r\ \approx)
%%    & \equiv 
%%    & (\Tset{T},\ \MapSet{r},\ \approx_{set}) \\ 
%% \mathrm{Reduce}(r',\ (T\ r\ \approx)) 
%%    & \equiv 
%%    & (\Tset{T},\ r' \circ r,\ \approx \circ (r' \times r')) \\ 
%% \end{array} 
%% \] 


%% \[
%% \begin{array}{rcl}     
%% X \times Y 
%%   & \equiv 
%%   & \{(x,\ y) \mid x \in X,\ y \in Y\} \\ 
%% X + Y 
%%   & \equiv 
%%   & \{\mathrm{inl}(x) \mid x \in X\} \cup \{\mathrm{inr}(y) \mid y \in Y\} 
%% \end{array} 
%% \] 

%% \[
%% \begin{array}{rcl}     
%% (f \times g)(x, y) 
%%   & \equiv 
%%   & (f(x), g(y)) \\ 
%% (f + g)(\mathrm{inl}(x))
%%   & \equiv 
%%   & f(x) \\ 
%% (f + g)(\mathrm{inr}(y))
%%   & \equiv 
%%   & g(y) \\ 
%% \end{array} 
%% \] 

%% Concrete types are easily implemented in traditional programming languages as a datatype. 
%% \[
%% \begin{array}{rcll} 
%% T & ::=  & \BB        & (\mbox{booleans}) \\ 
%%   & \mid & \NN        & (\mbox{natural numbers}) \\ 
%%   & \mid & \Sigma^*   & (\mbox{words over alphabet $\Sigma$}) \\ 
%%   & \mid & T \times t & (\mbox{product}) \\ 
%%   & \mid & T + T      & (\mbox{sum}) \\ 
%%   & \mid & \Tseq{T}   & (\mbox{lists over $T$}) \\ 
%%   & \mid & \Tset{T}   & (\mbox{finite sets over $T$}) \\ 
%% \end{array} 
%% \] 

%% \section{Base Semigroups}

%% \[
%% \begin{array}{|c|c|c|c|c|c|c|c|}\hline 
%% S    & \bullet         & \mbox{description}          & \alpha   & \omega & \propname{CM} & \propname{SL} & \propname{IP} 
%% \\ \hline 
%% S    & \opleft        & x \opleft y = x             & & &  & \star & \star \\
%% S    & \opright       & x \opright y = y            & & &  & \star & \star \\
%% S^*  & \cdot          & \mbox{concatenation}        & \epsilon &      &  &  & \\
%% S^+  & \cdot          & \mbox{concatenation}        &          &      &  &  & \\

%% \{t,\ f\}  & \wedge   & \mbox{conjunction}          & $t$     & $f$   &  \star &  \star & \star \\ 
%% \{t,\ f\}  & \vee     & \mbox{disjunction}          & $f$     & $t$   &  \star &  \star & \star \\

%% \NN  & \min           & \mbox{minimum}              &          & 0    &  \star &  \star & \star \\ 
%% \NN  & \max           & \mbox{maximum}              & 0        &      &  \star &  \star & \star \\

%% 2^W  & \cup           & \mbox{union}                & \{\}     & W    &  \star &  & \star \\
%% 2^W  & \cap           & \mbox{intersection}         & W        & \{\} &  \star &  & \star \\

%% \funfin(2^U)  & \cup & \mbox{union}            & \{\}     &     &  \star &  & \star \\
%% \funfin(2^U ) & \cap & \mbox{intersection}     &         & \{\} &  \star &  & \star \\


%% \NN  & +              & \mbox{addition}             & 0        &      &  \star &        & \\
%% \NN  & \times         & \mbox{multiplication}        & 1        & 0    &  \star &        & \\
%% \hline 
%% \end{array} 
%% \]

%% \begin{tabular}{l} 
%% $W$ a finite set, $U$ an infinite set. For set $Y$, $\funfin(Y) \equiv \{X \in Y\mid X \mbox{ is finite}\}$ 
%% \end{tabular} 


