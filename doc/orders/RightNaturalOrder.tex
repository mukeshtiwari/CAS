\documentclass[../Summary.tex]{subfiles}

\begin{document}

\subsubsection{Summary}

\[
\begin{array}{lcrcl} 
\propname{TRUE}
	& \Rightarrow
	& \propname{RX}(\mathrm{RNO}(S,\ \bullet)) 
    & \Leftrightarrow
    & \propname{IP}(S,\ \bullet)  \\
\propname{AS}(S,\ \bullet)
	& \Rightarrow
	& \propname{TR}(\mathrm{RNO}(S,\ \bullet))
    & \Leftrightarrow
    & \propname{TRUE} \\
\propname{CM}(S,\ \bullet)
	& \Rightarrow
	& \propname{AY}(\mathrm{RNO}(S,\ \bullet))
    & \Leftrightarrow
    & \propname{TRUE} \\ 
\propname{CM}(S,\ \bullet)
	& \Rightarrow
	& \propname{TO}(\mathrm{RNO}(S,\ \bullet))
    & \Leftrightarrow
    & \propname{SL}(S,\ \bullet) \\
\propname{TRUE}
	& \Rightarrow
	& \propname{BM}(\mathrm{RNO}(S,\ \bullet))
    & \Leftrightarrow
    & \propname{RAN}(S,\ \bullet) \\ 
\propname{TRUE}
	& \Rightarrow
	& \propname{TP}(\mathrm{RNO}(S,\ \bullet))
    & \Leftrightarrow
    & \propname{LID}(S,\ \bullet) \\ 
\end{array} 
\] 

For detailed proofs see the dual versions for the LNO combinator.

\end{document}
