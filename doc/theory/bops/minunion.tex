\documentclass[10pt]{article}

% packages 
\usepackage[english]{babel}
\usepackage{amsmath}   % see http://www.ctan.org/pkg/amsmath
\usepackage{amssymb}   % 
\usepackage{amsfonts}  % 
\usepackage{scrextend} % used for indentation

% my macros 
\newcommand{\mtrx}[1]{\ensuremath{\mathbf{#1}}}
\newcommand{\NN}{\mathbb{N}}
\newcommand{\BB}{\mathbb{B}}
\newcommand{\Id}{\mathrm{id}}
\newcommand{\Tseq}[1]{{\mathrm{seq}}({#1})}
\newcommand{\Tset}[1]{{\mathrm{set}}({#1})}
\newcommand{\MapSeq}[1]{{\mathrm{map}}({#1})}
\newcommand{\MapSet}[1]{{\mathrm{map}}({#1})}
\newcommand{\SemCarrier}[1]{{\mathrm{Carrier}}({#1})}
\newcommand{\propname}[1]{{\mathbb{#1}}}
\newcommand{\minunion}{\cup_{\min}^{\leq}}
\newcommand{\proof}{\vspace{1em} \textbf{Proof} \vspace{1em}}

\newenvironment{ind}[0]{\begin{addmargin}[1em]{0em}\vspace{0.5em}}{\end{addmargin}\vspace{0.5em}}

\DeclareMathOperator{\opleft}{left}
%\newcommand{\opleft}{\ensuremath{\mathrm{left}}}
%\newcommand{\opright}{\ensuremath{\mathrm{right}}}
\DeclareMathOperator{\opright}{right}

% Settings
\setlength{\parindent}{0pt}


%Document
\title{MinUnion results} 
\author{Timothy G. Griffin \& Matthew L. Daggitt} 

\begin{document}
\maketitle
\date 

\begin{abstract} 
The document presents some theoretical results for the $\minunion$ operator. 
\end{abstract} 


\section{Definition}

Let $\leq$ be a partial order on some set $S$ with at least two elements. 
For $X \subseteq S$, define 
\[
\min_\leq(X) \equiv \{x \in X \mid \forall y \in X,\  \neg(y < x)\}. 
\] 
Define 
\[
{\cal P}_{\mathrm{fin}}(S,\ \lesssim) 
   \equiv 
\{X \subseteq S\mid \mbox{$X$ finite and }\min_\leq(X) = X\} 
\] 
and 
\[
  X\ \minunion Y  \equiv  \min_\leq(X \cup Y) 
\] 
then
\[ 
\begin{array}{rcl} 
\mathrm{MinUnion}(S,\ \leq)
   & \equiv 
   & ({\cal P}_{\mathrm{fin}}(S,\ \leq),\ \minunion)
\end{array} 
\] 

\section{Summary}

The current pre-conditions required for generating necessary and sufficient conditions for each of the currently tracked properties are:
\begin{itemize}
\item $\propname{TR}(\mathrm{S, \leq})$ : required for associativity
\item $\propname{BM}(\mathrm{S, \leq})$ : required for annihilator
\item $\propname{AY}(\mathrm{S, \leq})$ : required for annihilator
\end{itemize}
Interestingly $\propname{RX}(\mathrm{S, \leq})$ is not required.

\section{Lemmas}

\subsection{Absorbing}

$\propname{TR}(\mathrm{S, \leq}) \Rightarrow \forall X Y : \min_\leq(X \cup \min_\leq(Y)) = \min_\leq(X \cup Y)$

\proof

Assume $z \in \min_\leq(X \cup Y)$. 
\begin{ind}
Then we have:
\begin{equation}
z \in X \ \vee \ z \in Y
\end{equation}
\begin{equation} \label{eq:lem_abs_minset_left_X}
\forall x \in X : \neg (x < z)
\end{equation}
\begin{equation} \label{eq:lem_abs_minset_left_Y}
\forall y \in Y : \neg (y < z)
\end{equation}

By Eq~\ref{eq:lem_abs_minset_left_Y} and the fact that $\min_\leq(Y) \subseteq Y$, we have that
\begin{equation*} 
\forall y \in \min_\leq(Y) : \neg (y < z)
\end{equation*}
and combining this with Eq~\ref{eq:lem_abs_minset_left_X} gives
\begin{equation} \label{eq:lem_abs_minset_left_XminsetY}
\forall x \in X \cup \min_\leq(Y) : \neg (x < z)
\end{equation}

Case $z \in X$
\begin{ind}
Clearly $z \in X \, \cup \, \min_\leq(Y)$ and by combining this with Eq~\ref{eq:lem_abs_minset_left_XminsetY} we have that $x \in \min_\leq(X \cup \min_\leq(Y))$.
\end{ind}

Case $z \in Y$
\begin{ind}
By Eq~\ref{eq:lem_abs_minset_left_Y} we have that $z \in \min_\leq(Y)$ and hence we have that $z \in X \cup \min_\leq(Y)$. By Eq \ref{eq:lem_abs_minset_left_XminsetY} we have that $z \in \min_\leq(X \cup \min_\leq(Y))$.
\end{ind}

Hence $z \in \min_\leq(X \cup Y) \Rightarrow z \in \min_\leq(X \cup \min_\leq(Y))$.
\end{ind}

\vspace{2em}

Assume $z \in \min_\leq(X \cup \min_\leq(Y))$.

\begin{ind}
Then we have:
\begin{equation}
z \in X \ \vee \ z \in \min_\leq(Y)
\end{equation}
\begin{equation} \label{eq:lem_abs_minset_right_X}
\forall x \in X : \neg (x < z)
\end{equation}
\begin{equation} \label{eq:lem_abs_minset_right_minsetY}
\forall y \in \min_\leq(Y) : \neg (y < z)
\end{equation}

Assume $\exists w \in Y : y < w$
\begin{ind}
Clearly $w \notin \min_\leq(Y)$ otherwise Eq~\ref{eq:lem_abs_minset_right_minsetY} would be contradicted. Therefore by definition of $\min_\leq$ and $\propname{TR}(\mathrm{S, \leq})$ there exists a $y \in \min_\leq(Y)$ such that $y < w$. But again by $\propname{TR}(\mathrm{S, \leq})$ this gives us $y < w < z$ which contradicts Eq~\ref{eq:lem_abs_minset_right_minsetY}.
\end{ind}
Hence
\begin{equation} \label{eq:lem_abs_minset_right_Y}
\forall y \in Y : \neg (y < z)
\end{equation}

Case $z \in X$
\begin{ind}
Then $z \in X \cup Y$, and by Eq~\ref{eq:lem_abs_minset_right_X} \& \ref{eq:lem_abs_minset_right_Y} we have that $\forall x \in X \cup Y : \neg (x < z)$. Therefore $z \in \min_\leq(X \cup Y)$.
\end{ind}

Case $z \in \min_\leq(Y)$
\begin{ind}
Then $z \in Y$ and hence $z \in X \cup Y$, and by Eq~\ref{eq:lem_abs_minset_right_X} \& \ref{eq:lem_abs_minset_right_Y} we have that $\forall x \in X \cup Y : \neg (x < z)$. Therefore $z \in \min_\leq(X \cup Y)$.
\end{ind}
Hence $z \in \min_\leq(X \cup \min_\leq(Y)) \Rightarrow z \in \min_\leq(X \cup Y)$.
\end{ind}


\vspace{1em}

Therefore $\min_\leq(X \cup \min_\leq(Y)) = \min_\leq(X \cup Y)$

\section{Properties}

\subsection{Associative}

$\propname{TR}(\mathrm{S, \leq}) \Rightarrow (\propname{AS}(\mathrm{MinUnion(S,\leq)}) \Leftrightarrow \propname{TRUE})$

\proof

Using the absorbing lemma we have:
\begin{align*}
X \minunion (Y \minunion Z) & = \min_\leq(X \cup \min_\leq(Y \cup Z)) \\
							& = \min_\leq(X \cup (Y \cup Z)) \\
							& = \min_\leq((X \cup Y) \cup Z) \\
							& = \min_\leq(\min_\leq(X \cup Y) \cup Z) \\
							& = (X \minunion Y) \minunion Z
\end{align*}



\subsection{Identity}

$\propname{ID}(\mathrm{MinUnion(S,\leq)}) \Leftrightarrow \propname{TRUE}$

\proof

Witness : $\emptyset$
\begin{align*}
X \minunion \emptyset 	& = \min_\leq(X \cup \emptyset) \\
						& = \min_\leq(X) \\
						& = X
\end{align*}

\begin{align*}
\emptyset \minunion X 	& = \min_\leq(\emptyset \cup X) \\
						& = \min_\leq(X) \\
						& = X
\end{align*}



\subsection{Annihilator}

$\propname{BM}(\mathrm{S,\leq}) \wedge \propname{AY}(\mathrm{S,\leq}) \Rightarrow (\propname{AN}(\mathrm{MinUnion(S,\leq)}) \Leftrightarrow \propname{TRUE})$

\proof

Witness: $\{\bot\}$

\begin{align*}
X \minunion \{\bot\} 	& = \min_\leq(X \cup \{\bot\}) \\
						& = \{\bot\}
\end{align*}

\begin{align*}
\{\bot\} \minunion X 	& = \min_\leq(\{\bot\} \cup X) \\
						& = \{\bot\}
\end{align*}



\subsection{Commutative}

$\propname{CM}(\mathrm{MinUnion(S,\leq)}) \Leftrightarrow \propname{TRUE}$

\proof
\begin{align*}
X \minunion Y 	& = \min_\leq(X \cup Y) \\
				& = \min_\leq(Y \cup X) \\
				& = Y \minunion X
\end{align*}



\subsection{Selective}

$\propname{SL}(\mathrm{MinUnion(S,\leq)}) \Leftrightarrow \propname{TO}(\mathrm{(S,\leq)})$

\proof

Assume $\propname{TO}(\mathrm{(S,\leq)})$.

\begin{ind}
Then every set can only contain a single element. Hence for all $x$ and $y$:
\begin{align*}
\{ x \} \minunion \{ y \} 	& = \min_\leq(\{ x \} \cup \{ y \}) \\
							& = \min_\leq(\{ x , y \}) 
\end{align*}
and as $x$ and $y$ are totally ordered, the result must either equal $\{ x \}$ or $\{ y \}$ and therefore we have that $\propname{SL}(\mathrm{MinUnion(S,\leq)})$.
\end{ind}

Assume $\neg \propname{TO}(\mathrm{(S,\leq)})$.

\begin{ind}
Then there exists $x$ and $y$ such that $x \nleq y$ and $y \nleq x$.
\begin{align*}
\{ x \} \minunion \{ y \} 	& = \min_\leq(\{ x \} \cup \{ y \}) \\
							& = \min_\leq(\{ x , y \}) \\
							& = \{ x , y \}
\end{align*}
Hence we have that $\neg \propname{SL}(\mathrm{MinUnion(S,\leq)})$.
\end{ind}



\subsection{Idempotent}
$\propname{IP}(\mathrm{MinUnion(S,\leq)}) \Leftrightarrow \propname{TRUE}$

\proof

\begin{align*}
X \minunion X 	& = \min_\leq(X \cup X) \\
				& = \min_\leq(X) \\
				& = X
\end{align*}



\subsection{IsLeft}
$\propname{IL}(\mathrm{MinUnion(S,\leq)}) \Leftrightarrow \propname{FALSE}$

\proof

There exists an identity for $\mathrm{MinUnion(S,\leq)}$, therefore $\neg \propname{IL}(\mathrm{MinUnion(S,\leq)})$.



\subsection{IsRight}
$\propname{IR}(\mathrm{MinUnion(S,\leq)}) \Leftrightarrow \propname{FALSE}$

\proof

There exists an identity for $\mathrm{MinUnion(S,\leq)}$, therefore $\neg \propname{IR}(\mathrm{MinUnion(S,\leq)})$.



\subsection{LeftCancellative}
$\propname{LC}(\mathrm{MinUnion(S,\leq)}) \Leftrightarrow \propname{FALSE}$

\proof

There exists an identity for $\mathrm{MinUnion(S,\leq)}$, and at least two elements in $S$ and therefore two non-identity elements in ${\cal P}_{\mathrm{fin}}(S,\ \lesssim)$. Therefore $\neg \propname{LC}(\mathrm{MinUnion(S,\leq)})$.

\subsection{RightCancellative}
$\propname{RC}(\mathrm{MinUnion(S,\leq)}) \Leftrightarrow \propname{FALSE}$

\proof

There exists an identity for $\mathrm{MinUnion(S,\leq)}$, and at least two elements in $S$ and therefore two non-identity elements in ${\cal P}_{\mathrm{fin}}(S,\ \lesssim)$. Therefore $\neg \propname{RC}(\mathrm{MinUnion(S,\leq)})$.



\subsection{LeftConstant}
$\propname{LK}(\mathrm{MinUnion(S,\leq)}) \Leftrightarrow \propname{FALSE}$

\proof

There exists an identity for $\mathrm{MinUnion(S,\leq)}$, therefore $\neg \propname{LK}(\mathrm{MinUnion(S,\leq)})$.



\subsection{RightConstant}
$\propname{RK}(\mathrm{MinUnion(S,\leq)}) \Leftrightarrow \propname{FALSE}$

\proof

There exists an identity for $\mathrm{MinUnion(S,\leq)}$, therefore $\neg \propname{RK}(\mathrm{MinUnion(S,\leq)})$.



\subsection{AntiLeft}
$\propname{AL}(\mathrm{MinUnion(S,\leq)}) \Leftrightarrow \propname{FALSE}$

\proof

There exists an identity for $\mathrm{MinUnion(S,\leq)}$, therefore $\neg \propname{AL}(\mathrm{MinUnion(S,\leq)})$.


\subsection{AntiRight}
$\propname{AR}(\mathrm{MinUnion(S,\leq)}) \Leftrightarrow \propname{FALSE}$

\proof

There exists an identity for $\mathrm{MinUnion(S,\leq)}$, therefore $\neg \propname{AR}(\mathrm{MinUnion(S,\leq)})$.

\end{document}